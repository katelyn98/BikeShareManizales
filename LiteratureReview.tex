\documentclass[12pt]{article}
\usepackage[margin=1in]{geometry}
\usepackage[nottoc]{tocbibind}
\usepackage{setspace}
\usepackage{indentfirst}

%\title{Adults' Perspectives on the Current Bike Sharing System in Manizales, Colombia}

\doublespacing

\begin{document}
%\maketitle

\section*{Literature Review}
Urban planners are constantly deriving and evaluating new strategies to satisfy desires for futuristic and sustainable 
transportation systems. Some of the most successful transportation initiatives are emerging from wealthy and developed 
European cities such as Amsterdam, Copenhagen, and Stockholm \cite{DeliotteReport}. Culture and infrastructure around transportation 
within these European cities set the status quo for biking to be a reliable mode of transportation \cite{DeliotteReport}.  A common characteristic 
that distinguishes cities with strong transportation systems is their initiatives to grow as a smart city that 
fosters sustainable development. A smart city is composed of several components, attributes, and themes; this research study 
specifically focuses on the transportation component of a smart city, also known as smart transportation \cite{DefiningSmartCity}. 
Smart transportation, or smart mobility, in the context of this research study will be defined as transportation that is offered on demand
and is presented as efficient, sustainable, flexible, and eco-friendly \cite{SmartTransportation}. 

Improving transportation systems’ design and infrastructure can lead to be an 
overwhelming task accompanied with an exorbitant cost but in the long run the benefits will distinctly present themselves. 
Smart transportation initiatives specifically relating to biking and the market of bike sharing systems are growing rapidly within
cities. Some cities, as Deliotte points out, are lacking in infrastructure to foster a well-established commuter population comprised of 
bikers \cite{DeliotteReport}. Deliotte acknowledges from a study they did on smart mobility across cities in the United States that, “slightly 
more than a quarter of current commuters could switch to bike commuting as one of their main modes of commuting if barriers 
to biking were substantially reduced” \cite{DeliotteReport}. This finding presents an opportunity to find out how adults view the current infrastructure
set up for bikes in Manizales, Colombia, and what attributes are influencing their assessment of the current bike sharing system there.

\subsection*{Manizales, Colombia \& Transportation}
Manizales, Colombia is the capital city in Caldas, Colombia with a record population of 400,436 people in 2018 from their Census with
71\% of the population being between 15 and 64 years old \cite{CalidaddeVida}. The city of Manizales is topographically very mountainous 
naturally presenting barriers that need to be considered when planning for a successful smart mobility presence. To provide a strong case
to implement smart mobility in Manizales, Colombia, it is necessary to understand the current transportation cutlure. In 2018, it was 
recorded that there was 445 vehicles per 1000 people which is equivalent of saying that there was one vehicle per every two people \cite{CalidaddeVida}.
Despite the report of 445 vehicles per 100 people, another model shows that 56\% of the population's main mode of transportation is by bus \cite{CalidaddeVida}. 
One of the benefits of living within city limits include the opportunity to access public trasnportation and be closer to everything which makes
owning a car unnecessary. This model however points out the only 12\% of the population reported their main mode of transportation to be walking or biking.     
\textit{I would like to add more information here as I keep working on it.}

\subsection*{Bike Sharing Systems in Manizales, Colombia}
Ever since the United Nations released the seventeen sustainable development goals (SDGs) for the world, countries have been consistently integrating
these goals into initiatives and projects within their cities. Manizales has been working on 
smart mobility initiatives for a few years now, but only recently have they been making significant progress. Their projects and initiatives for smart mobility 
are targeted towards advancing eleven of the seventeen SDGs \cite{OficinaDeLaBici}. The bike sharing system in Manizales, Colombia is particularly interesting 
because a company called CityBioBike has developed a biometric loaning system for bike renting \cite{CityBioBike}. CityBioBike partners with the Bike Office at 
the Universidad Católica de Manizales to provide a bike sharing system to the city \cite{OficinaDeLaBiciHome}.  

\subsection*{Factors that play into assessing a bike sharing system}

\subsection*{Future Improvments to bike sharing systems in general}

\bibliographystyle{unsrt}
\bibliography{references}

\end{document}