\documentclass[a4paper,10pt,BCOR10mm,oneside,headsepline]{scrartcl}
\usepackage[ngerman]{babel}
\usepackage[utf8]{inputenc}
\usepackage{wasysym}% provides \ocircle and \Box
\usepackage{enumitem}% easy control of topsep and leftmargin for lists
\usepackage{color}% used for background color
\usepackage{forloop}% used for \Qrating and \Qlines
\usepackage{ifthen}% used for \Qitem and \QItem
\usepackage{typearea}
\usepackage{tabularcalc}
\usepackage{geometry}
\usepackage[table]{xcolor}
\areaset{17cm}{26cm}
\setlength{\topmargin}{-1cm}
\usepackage{scrpage2}
\pagestyle{scrheadings}
\ihead{Perspectives on Smart Mobility Initiatives in Manizales, Colombia}
\ohead{\pagemark}
\chead{}
\cfoot{}

%%%%%%%%%%%%%%%%%%%%%%%%%%%%%%%%%%%%%%%%%%%%%%%%%%%%%%%%%%%%
%% Beginning of questionnaire command definitions %%
%%%%%%%%%%%%%%%%%%%%%%%%%%%%%%%%%%%%%%%%%%%%%%%%%%%%%%%%%%%%
%%
%% 2010, 2012 by Sven Hartenstein
%% mail@svenhartenstein.de
%% http://www.svenhartenstein.de
%%
%% Please be warned that this is NOT a full-featured framework for
%% creating (all sorts of) questionnaires. Rather, it is a small
%% collection of LaTeX commands that I found useful when creating a
%% questionnaire. Feel free to copy and adjust any parts you like.
%% Most probably, you will want to change the commands, so that they
%% fit your taste.
%%
%% Also note that I am not a LaTeX expert! Things can very likely be
%% done much more elegant than I was able to. If you have suggestions
%% about what can be improved please send me an email. I intend to
%% add good tipps to my website and to name contributers of course.
%%
%% 10/2012: Thanks to karathan for the suggestion to put \noindent
%% before \rule!

%% \Qq = Questionaire question. Oh, this is just too simple. It helps
%% making it easy to globally change the appearance of questions.
\newcommand{\Qq}[1]{\textbf{#1}}

%% \QO = Circle or box to be ticked. Used both by direct call and by
%% \Qrating and \Qlist.
\newcommand{\QO}{$\Box$}% or: $\ocircle$

%% \Qrating = Automatically create a rating scale with NUM steps, like
%% this: 0--0--0--0--0.
\newcounter{qr}
\newcommand{\Qrating}[1]{\QO\forloop{qr}{1}{\value{qr} < #1}{---\QO}}

%% \Qline = Again, this is very simple. It helps setting the line
%% thickness globally. Used both by direct call and by \Qlines.
\newcommand{\Qline}[1]{\noindent\rule{#1}{0.6pt}}

%% \Qlines = Insert NUM lines with width=\linewith. You can change the
%% \vskip value to adjust the spacing.
\newcounter{ql}
\newcommand{\Qlines}[1]{\forloop{ql}{0}{\value{ql}<#1}{\vskip0em\Qline{\linewidth}}}

%% \Qlist = This is an environment very similar to itemize but with
%% \QO in front of each list item. Useful for classical multiple
%% choice. Change leftmargin and topsep accourding to your taste.
\newenvironment{Qlist}{%
\renewcommand{\labelitemi}{\QO}
\begin{itemize}[leftmargin=1.5em,topsep=-.5em]
}{%
\end{itemize}
}

%% \Qtab = A "tabulator simulation". The first argument is the
%% distance from the left margin. The second argument is content which
%% is indented within the current row.
\newlength{\qt}
\newcommand{\Qtab}[2]{
\setlength{\qt}{\linewidth}
\addtolength{\qt}{-#1}
\hfill\parbox[t]{\qt}{\raggedright #2}
}

%% \Qitem = Item with automatic numbering. The first optional argument
%% can be used to create sub-items like 2a, 2b, 2c, ... The item
%% number is increased if the first argument is omitted or equals 'a'.
%% You will have to adjust this if you prefer a different numbering
%% scheme. Adjust topsep and leftmargin as needed.
\newcounter{itemnummer}
\newcommand{\Qitem}[2][]{% #1 optional, #2 notwendig
\ifthenelse{\equal{#1}{}}{\stepcounter{itemnummer}}{}
\ifthenelse{\equal{#1}{a}}{\stepcounter{itemnummer}}{}
\begin{enumerate}[topsep=2pt,leftmargin=2.8em]
\item[\textbf{\arabic{itemnummer}#1.}] #2
\end{enumerate}
}

%% \QItem = Like \Qitem but with alternating background color. This
%% might be error prone as I hard-coded some lengths (-5.25pt and
%% -3pt)! I do not yet understand why I need them.
\definecolor{bgodd}{rgb}{0.8,0.8,0.8}
\definecolor{bgeven}{rgb}{0.9,0.9,0.9}
\newcounter{itemoddeven}
\newlength{\gb}
\newcommand{\QItem}[2][]{% #1 optional, #2 notwendig
\setlength{\gb}{\linewidth}
\addtolength{\gb}{-5.25pt}
\ifthenelse{\equal{\value{itemoddeven}}{0}}{%
\noindent\colorbox{bgeven}{\hskip-3pt\begin{minipage}{\gb}\Qitem[#1]{#2}\end{minipage}}%
\stepcounter{itemoddeven}%
}{%
\noindent\colorbox{bgodd}{\hskip-3pt\begin{minipage}{\gb}\Qitem[#1]{#2}\end{minipage}}%
\setcounter{itemoddeven}{0}%
}
}

%%%%%%%%%%%%%%%%%%%%%%%%%%%%%%%%%%%%%%%%%%%%%%%%%%%%%%%%%%%%
%% End of questionnaire command definitions %%
%%%%%%%%%%%%%%%%%%%%%%%%%%%%%%%%%%%%%%%%%%%%%%%%%%%%%%%%%%%%

\begin{document}
\renewcommand{\QO}{$\ocircle$}% Use circles now instead of boxes.
%%\begin{center}
%%\textbf{\huge Questionnaire \LaTeX}
%%\end{center}\vskip1em

\noindent Thank you for participating in this study; your responses and honesty are appreciated.
Your responses to the questions below will remain annonynous and confidential.
You do not have to answer any question that you do not want to. This study is not
associated with any public or private organizations. Analysis of data will be for
academic use only at the University of Pittsburgh (EE. UU). 

%\noindent \textit{Please note that no tabular environment is used in
%this example questionnaire. Of course, you could use tabular to
%create more complex layout.}

\subsection*{About you}

\Qitem{ \Qq{Please select your age range: }\hskip0.4cm \QO{}
18-28 \hskip0.4cm \QO{} 29-40 \hskip0.4cm \QO{} 41-50 \hskip0.4cm \QO{} 51-60 \hskip0.4cm \QO{} 61+}

\Qitem{ \Qq{Please select the gender that you identify with: }\hskip0.4cm \QO{}
Female \hskip0.4cm \QO{} Male \hskip0.4cm Other: \Qline{2.5cm} }

\Qitem{ \Qq{Rate your physical activity level based on approx. steps/day: }} \hskip1cm \QO{}
1.000 - 5.000 (not active) \hskip0.4cm \QO{} 5.000 - 10.000 (active) \hskip0.4cm 10.0000 + (very active)

\renewcommand{\QO}{$\Box$}
\Qitem{ \Qq{What adjectives do you associate with the statement: "Transportation in Manizales, Colombia" ?}
    \begin{Qlist}
    \item[]\QO{} Outdated \hskip1cm \QO{} Technologically Advanced \hskip1cm \QO{} Unsafe travel conditions \hskip1cm \QO{} Consistent
    \item[]\QO{} Not reliable \hskip0.7cm \QO{} Complicated \hskip0.7cm \QO{} Diverse \hskip0.6cm \QO{} Expensive \hskip0.6cm \QO{} Enforced Rules
    \item[]\QO{} Overcrowded \hskip0.5cm \QO{} Not easily accessible \hskip1.8cm \QO{} Other: \Qline{5cm}
    \end{Qlist}
}

\renewcommand{\QO}{$\ocircle$}
\Qitem{ \Qq{Do you own or share ownership of a vehicle?}
    \begin{Qlist}
    \item Yes, I personally own a vehicle.
    \item No, I do not personally own a vehicle.
    \item Not sure.
    \end{Qlist}
\textit{If you checked 'No' or 'Not sure', proceed to question 8.}
}

\renewcommand{\QO}{$\Box$}
\Qitem{ \Qq{What kind of personal vehicle(s) do you own/share ownership?}
    \begin{Qlist}
    \item[]\QO{} Car \hskip4.2cm \QO{} Electric-Powered Scooter \hskip1.5cm \QO{} Electric-Powered Bicycle
    \item[]\QO{} Non-Electric Scooter \hskip1.5cm \QO{} Pedal Bicycle (Non-Eletric) \hskip1.25cm \QO{} Motorcycle
    \item[]\QO{} Other: \Qline{6cm}
    \end{Qlist}
}

\subsection*{Current Smart Mobility Initiatives in Manizales, Colombia}

\renewcommand{\QO}{$\ocircle$}
\Qitem{ \Qq{Do you use any public or private modes of transportation?}
\hskip1cm\QO{} yes \hskip1cm \QO{} no}{}

\renewcommand{\QO}{$\Box$}
\Qitem{{If you answered no to Q8: Please identify what reasons most closely align with why you do not use 
any public or private modes or transportation.}
    \begin{Qlist}
    \item[]\QO{} Not close to where I live/work \hskip1.5cm \QO{} Long travel times  \hskip1cm \QO{} Not envrionmentally friendly
    \item[]\QO{} Not offered when I need to travel \hskip0.8cm \QO{}I like/prefer to walk  \hskip0.95cm \QO{} I don't need to
    \item[]\QO{} Other: \Qline{6cm}
    \end{Qlist}

}
\Qitem{{If you answered yes to Q8: Which mode(s) of public or private transportation do you use at least once every week?} 
    \begin{Qlist}
    \item[]\QO{} Taxi \hskip4.21cm \QO{} Public Bus/Buseta \hskip3.1cm \QO{} Bike-Sharing Program
    \item[]\QO{} Ride-Sharing Program \hskip1.5cm \QO{} Carpooling with friends/family \hskip1cm \QO{} Cable Aereo
    \item[] \QO{} Other: \Qline{6cm}
    \end{Qlist}
}

\Qitem{{If you answered yes to Q8: Which mode(s) of public or private transportation do you dislike using or try to avoid?} 
    \begin{Qlist}
    \item[]\QO{} Taxi \hskip4.21cm \QO{} Public Bus/Buseta \hskip3.1cm \QO{} Bike-Sharing Program
    \item[]\QO{} Ride-Sharing Program \hskip1.5cm \QO{} Carpooling with friends/family \hskip1cm \QO{} Cable Aereo
    \item[] \QO{} Other: \Qline{6cm}
    \end{Qlist}
}

\vskip.5em
\textbf{Consider the modes of transportation you do not use in your daily life or try to avoid. Please identify below
which factors strongly influence why you don't use certain modes of transportation.}\\

\QItem[a]{ \Qq{Travel times are too long} \Qtab{7.5cm}{strongly disagree \Qrating{5} strongly agree}}

\QItem[b]{ \Qq{Stress inducing} \Qtab{7.5cm}{strongly disagree \Qrating{5} strongly agree}}\\
\textbf{Please continue on to the back of this page to finish answering survey questions.}

\QItem[c]{ \Qq{Not technologically advanced} \Qtab{7.5cm}{strongly disagree \Qrating{5} strongly agree}}

\QItem[d]{ \Qq{Not available when I need it}\Qtab{7.5cm}{strongly disagree \Qrating{5} strongly agree}}

\QItem[e]{ \Qq{Breaks down a lot}\Qtab{7.5cm}{strongly disagree \Qrating{5} strongly agree}}

\QItem[f]{ \Qq{Too crowded}\Qtab{7.5cm}{strongly disagree \Qrating{5} strongly agree}}

\QItem[g]{ \Qq{Not clean}\Qtab{7.5cm}{strongly disagree \Qrating{5} strongly agree}}

\QItem[h]{ \Qq{Too expensive}\Qtab{7.5cm}{strongly disagree \Qrating{5} strongly agree}}

\QItem[i]{ \Qq{Complicated registration process}\Qtab{7.5cm}{strongly disagree \Qrating{5} strongly agree}}

\QItem[j]{ \Qq{I need to exercise}\Qtab{7.5cm}{strongly disagree \Qrating{5} strongly agree}}

\QItem[k]{ \Qq{Not easily accessible}\Qtab{7.5cm}{strongly disagree \Qrating{5} strongly agree}}\\

\Qitem{Do you have any other comments about why you don't use certain modes of transportation? Provide examples and 
the specific mode of transportation that you are referring to. } \Qline{17cm}\\ \Qline{17cm}\\ \Qline{17cm}
\Qitem{What factors are positively and/or negatively impacted by the modes of transportation you use? Only fill out
the modes of transportation that apply to you.} 
SL = Stress-level \hskip1cm PF = Physical Fitness \hskip1cm FS = Financial Savings \hskip1cm WL = Work-Life Balance

{\rowcolors{1}{white!80!black!20}{white!70!black!60}
\begin{center}
\begin{tabular}{ | m{10em} | m{5cm}| m{5cm} | } 
\hline
 & Positively Impacts & Negatively Impacts \\ 
\hline
Personal Motor Vehicle & \QO SL\hskip0.5cm \QO PF \hskip0.5cm \QO FS \hskip0.5cm \QO WL 
     & \QO SL\hskip0.5cm \QO PF \hskip0.5cm \QO FS \hskip0.5cm \QO WL \\ 
\hline
Personal Bike & \QO SL\hskip0.5cm \QO PF \hskip0.5cm \QO FS \hskip0.5cm \QO WL 
& \QO SL\hskip0.5cm \QO PF \hskip0.5cm \QO FS \hskip0.5cm \QO WL \\ 
\hline
Bike Sharing & \QO SL\hskip0.5cm \QO PF \hskip0.5cm \QO FS \hskip0.5cm \QO WL 
& \QO SL\hskip0.5cm \QO PF \hskip0.5cm \QO FS \hskip0.5cm \QO WL \\ 
\hline
Ride-Sharing & \QO SL\hskip0.5cm \QO PF \hskip0.5cm \QO FS \hskip0.5cm \QO WL 
& \QO SL\hskip0.5cm \QO PF \hskip0.5cm \QO FS \hskip0.5cm \QO WL \\ 
\hline
Taxi & \QO SL\hskip0.5cm \QO PF \hskip0.5cm \QO FS \hskip0.5cm \QO WL 
& \QO SL\hskip0.5cm \QO PF \hskip0.5cm \QO FS \hskip0.5cm \QO WL \\ 
\hline
Public Bus& \QO SL\hskip0.5cm \QO PF \hskip0.5cm \QO FS \hskip0.5cm \QO WL 
& \QO SL\hskip0.5cm \QO PF \hskip0.5cm \QO FS \hskip0.5cm \QO WL \\ 
\hline
Cable Aereo & \QO SL\hskip0.5cm \QO PF \hskip0.5cm \QO FS \hskip0.5cm \QO WL 
& \QO SL\hskip0.5cm \QO PF \hskip0.5cm \QO FS \hskip0.5cm \QO WL \\ 
\hline
Other: \Qline{2cm} & \QO SL\hskip0.5cm \QO PF \hskip0.5cm \QO FS \hskip0.5cm \QO WL 
& \QO SL\hskip0.5cm \QO PF \hskip0.5cm \QO FS \hskip0.5cm \QO WL \\ 
\hline
\end{tabular}
\end{center}

\subsection*{Potential Smart Mobility Initiatives in Manizales, Colombia}

If the below features were a part of tranportation in Manizales, Colombia, high likely are you to use it? \\

\QItem[a]{ \Qq{Free wifi on Cable Aereo and public buses} \Qtab{7.5cm}{very unlikely \Qrating{5} very likely}}

\QItem[b]{ \Qq{Complete streets} \Qtab{7.5cm}{very unlikely \Qrating{5} very likely}}\\
Complete streets are streets that separate motorvehicles from bicyles and from pedestrians by creating separate lanes for each mode of transportation.

\QItem[c]{ \Qq{Protected bike-only lane} \Qtab{7.5cm}{very unlikely \Qrating{5} very likely}}

\QItem[d]{ \Qq{Smart Phone Apps} \Qtab{7.5cm}{very unlikely \Qrating{5} very likely}}

\QItem[e]{ \Qq{Smart Phone charging ports} \Qtab{7.5cm}{very unlikely \Qrating{5} very likely}}

\Qitem{{For the questions between 14a. and 14e. that you answered very unlikely or unlikely, please select why below.}
    \begin{Qlist}
    \item[]\QO{} Not necessary \hskip2.5cm \QO{} Doesn't apply to me \hskip2cm \QO{} I'm not good with technology 
    \item[]\QO{} Not safe \hskip3.3cm \QO{} Not practical  \hskip3.2cm \QO{} I don't own or ride a bike
    \item[] \QO{} Other: \Qline{6cm}
    \end{Qlist}
}
\Qitem{{For the questions between 14a. and 14e. that you answered very likely or likely, please select why below.}
    \begin{Qlist}
    \item[]\QO{} Improves efficiency in commute \hskip0.5cm \QO{} Decreases stress \hskip2.5cm \QO{} Envrionmentally friendly
    \item[]\QO{}  Improves my personal safety \hskip1cm \QO{} Increases my physical fitness 
    \item[]\QO{} Other: \Qline{6cm}
    \end{Qlist}
}

\Qitem{Do you have any other comments about the current or potential transportation and it's impact on your quality of life? Write them in the space below.}

\end{document}