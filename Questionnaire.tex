\documentclass[a4paper,10pt,BCOR10mm,oneside,headsepline]{scrartcl}
\usepackage[ngerman]{babel}
\usepackage[utf8]{inputenc}
\usepackage{wasysym}% provides \ocircle and \Box
\usepackage{enumitem}% easy control of topsep and leftmargin for lists
\usepackage{color}% used for background color
\usepackage{forloop}% used for \Qrating and \Qlines
\usepackage{ifthen}% used for \Qitem and \QItem
\usepackage{typearea}
\usepackage{tabularcalc}
\usepackage{geometry}
\usepackage[table]{xcolor}
\areaset{17cm}{26cm}
\usepackage{scrpage2}


%%Question 8 5-point scale strongly agree and disagree and neutral 
%%scale based on the factor to not be biased
%%re-word question 8 select one option for each line
%get rid of question 6
%% what do you consider the most important for bike transportaiton specifically (ask after the best and worst)
%%what benefits 
%%what areas could be improved
%%which areas do they do the best and which are they worst
%% get rid of 11b
%%change 12 to be a rated of safe, practical, interested, comfortable. 
%%question 14 change suggestions and recommendations of initiatves. 

%%%%%%%%%%%%%%%%%%%%%%%%%%%%%%%%%%%%%%%%%%%%%%%%%%%%%%%%%%%%
%% Beginning of questionnaire command definitions %%
%%%%%%%%%%%%%%%%%%%%%%%%%%%%%%%%%%%%%%%%%%%%%%%%%%%%%%%%%%%%
%%
%% 2010, 2012 by Sven Hartenstein
%% mail@svenhartenstein.de
%% http://www.svenhartenstein.de
%%
%% Please be warned that this is NOT a full-featured framework for
%% creating (all sorts of) questionnaires. Rather, it is a small
%% collection of LaTeX commands that I found useful when creating a
%% questionnaire. Feel free to copy and adjust any parts you like.
%% Most probably, you will want to change the commands, so that they
%% fit your taste.
%%
%% Also note that I am not a LaTeX expert! Things can very likely be
%% done much more elegant than I was able to. If you have suggestions
%% about what can be improved please send me an email. I intend to
%% add good tipps to my website and to name contributers of course.
%%
%% 10/2012: Thanks to karathan for the suggestion to put \noindent
%% before \rule!

%% \Qq = Questionaire question. Oh, this is just too simple. It helps
%% making it easy to globally change the appearance of questions.
\newcommand{\Qq}[1]{\textbf{#1}}

%% \QO = Circle or box to be ticked. Used both by direct call and by
%% \Qrating and \Qlist.
\newcommand{\QO}{$\Box$}% or: $\ocircle$

%% \Qrating = Automatically create a rating scale with NUM steps, like
%% this: 0--0--0--0--0.
\newcounter{qr}
\newcommand{\Qrating}[1]{\QO\forloop{qr}{1}{\value{qr} < #1}{---\QO}}

%% \Qline = Again, this is very simple. It helps setting the line
%% thickness globally. Used both by direct call and by \Qlines.
\newcommand{\Qline}[1]{\noindent\rule{#1}{0.6pt}}

%% \Qlines = Insert NUM lines with width=\linewith. You can change the
%% \vskip value to adjust the spacing.
\newcounter{ql}
\newcommand{\Qlines}[1]{\forloop{ql}{0}{\value{ql}<#1}{\vskip0em\Qline{\linewidth}}}

%% \Qlist = This is an environment very similar to itemize but with
%% \QO in front of each list item. Useful for classical multiple
%% choice. Change leftmargin and topsep accourding to your taste.
\newenvironment{Qlist}{%
\renewcommand{\labelitemi}{\QO}
\begin{itemize}[leftmargin=1.5em,topsep=-.5em]
}{%
\end{itemize}
}

%% \Qtab = A "tabulator simulation". The first argument is the
%% distance from the left margin. The second argument is content which
%% is indented within the current row.
\newlength{\qt}
\newcommand{\Qtab}[2]{
\setlength{\qt}{\linewidth}
\addtolength{\qt}{-#1}
\hfill\parbox[t]{\qt}{\raggedright #2}
}

%% \Qitem = Item with automatic numbering. The first optional argument
%% can be used to create sub-items like 2a, 2b, 2c, ... The item
%% number is increased if the first argument is omitted or equals 'a'.
%% You will have to adjust this if you prefer a different numbering
%% scheme. Adjust topsep and leftmargin as needed.
\newcounter{itemnummer}
\newcommand{\Qitem}[2][]{% #1 optional, #2 notwendig
\ifthenelse{\equal{#1}{}}{\stepcounter{itemnummer}}{}
\ifthenelse{\equal{#1}{a}}{\stepcounter{itemnummer}}{}
\begin{enumerate}[topsep=2pt,leftmargin=2.8em]
\item[\textbf{\arabic{itemnummer}#1.}] #2
\end{enumerate}
}

%% \QItem = Like \Qitem but with alternating background color. This
%% might be error prone as I hard-coded some lengths (-5.25pt and
%% -3pt)! I do not yet understand why I need them.
\definecolor{bgodd}{rgb}{0.8,0.8,0.8}
\definecolor{bgeven}{rgb}{0.9,0.9,0.9}
\newcounter{itemoddeven}
\newlength{\gb}
\newcommand{\QItem}[2][]{% #1 optional, #2 notwendig
\setlength{\gb}{\linewidth}
\addtolength{\gb}{-5.25pt}
\ifthenelse{\equal{\value{itemoddeven}}{0}}{%
\noindent\colorbox{bgeven}{\hskip-3pt\begin{minipage}{\gb}\Qitem[#1]{#2}\end{minipage}}%
\stepcounter{itemoddeven}%
}{%
\noindent\colorbox{bgodd}{\hskip-3pt\begin{minipage}{\gb}\Qitem[#1]{#2}\end{minipage}}%
\setcounter{itemoddeven}{0}%
}
}

%%%%%%%%%%%%%%%%%%%%%%%%%%%%%%%%%%%%%%%%%%%%%%%%%%%%%%%%%%%%
%% End of questionnaire command definitions %%
%%%%%%%%%%%%%%%%%%%%%%%%%%%%%%%%%%%%%%%%%%%%%%%%%%%%%%%%%%%%

\begin{document}
\renewcommand{\QO}{$\ocircle$}% Use circles now instead of boxes.
%%\begin{center}
%%\textbf{\huge Questionnaire \LaTeX}
%%\end{center}\vskip1em

\noindent Thank you for participating in this study; your responses and honesty are appreciated.
Your responses to the questions below will remain annonynous and confidential.
You do not have to answer any question that you do not want to. This study is not
associated with any public or private organizations. Analysis of data will be for
academic use only at the University of Pittsburgh (EE. UU.). 

%\noindent \textit{Please note that no tabular environment is used in
%this example questionnaire. Of course, you could use tabular to
%create more complex layout.}

\subsection*{About you}

\Qitem{ \Qq{Please select your age range: }\hskip0.4cm \QO{}
18-28 \hskip0.4cm \QO{} 29-40 \hskip0.4cm \QO{} 41-50 \hskip0.4cm \QO{} 51-60 \hskip0.4cm \QO{} 61+}

\vspace{0.1cm}

\Qitem{ \Qq{Please select the gender that you identify with: }\hskip0.4cm \QO{}
Female \hskip0.4cm \QO{} Male \hskip0.4cm Other: \Qline{2.5cm} }

\vspace{0.1cm}


\Qitem{ \Qq{Have you ever heard of the free bike loan system called CityBioBike?}
\QO{} yes \hskip0.2cm \QO{} no \hskip0.2cm \QO{} not sure}{}

\vspace{0.1cm}

\Qitem{\Qq{Do you live within a 10 minute walk from a CityBioBike station?}
\QO{} yes \hskip0.5cm \QO{} no \hskip0.5cm \QO{} not sure}{}

\vspace{0.1cm}

\Qitem{ \Qq{Have you ever used CityBioBike? (If no, go to question 12.)}
\hskip1.5cm\QO{} yes \hskip1cm \QO{} no}{}
\Qitem{\Qq{Are you a frequent user? (If no, go to question 9.)}
\hskip1cm\QO{} yes \hskip1cm \QO{} no}{}

\renewcommand{\QO}{$\Box$}
\Qitem{\Qq{What are the main reasons you use CityBioBike? Select all that apply.} 
    \begin{Qlist}
    \item[]\QO{} To grocery shop \hskip1.5cm \QO{} To go to work \hskip1.5cm \QO{} To exercise \hskip1.5cm \QO{} To save money
    \item[]\QO{} To go to school \hskip1.5cm \QO{} To visit friends and family \hskip1cm \QO{} To enjoy the nature and mountains
    \item[] \QO{} Other: \Qline{6cm}
    \end{Qlist}
}
\renewcommand{\QO}{$\ocircle$}
\Qitem{ \Qq{How many times do you use CityBioBike on avg in a week?: }\hskip0.2cm \QO{}
0-4 \hskip0.2cm \QO{} 5-8 \hskip0.2cm \QO{} 9-12 \hskip0.2cm \QO{} 13+}

\vspace{0.3cm}

\subsection*{Current Transportation Initiatives in Manizales, Colombia}

\renewcommand{\QO}{$\Box$}

\vspace{0.1cm}

\textbf{9. Please rate how CityBioBike is doing with respect to each category listed below.} \\
\\
\QItem[a]{ \Qq{Cleanliness} \hskip5.7cm {not clean\Qrating{5} very clean}}

\QItem[b]{ \Qq{Registration Process} \hskip3.9cm {very hard \Qrating{5} very easy}}

\QItem[c]{ \Qq{Quality} \hskip5.9cm {breaks a lot \Qrating{5} rarely breaks}}

\QItem[d]{ \Qq{Bike availability}\hskip4.1cm {never available \Qrating{5} always available}}

\QItem[e]{ \Qq{Safety}\hskip6.9cm {not safe \Qrating{5} very safe}}

\QItem[f]{ \Qq{Bike Station Locations}\hskip2cm {needs more locations \Qrating{5} perfect as is}}

\vspace{0.1cm}

\Qitem{Do you have any other comments about why you do or don't like the bike sharing system in Manizlaes, Colombia? Provide examples. } \Qline{17cm}\\ \Qline{17cm}\\ \Qline{17cm}
%\Qitem{What factors are positively and/or negatively impacted by the modes of transportation you use? Only fill out
%the modes of transportation that apply to you.} 
%SL = Stress-level \hskip1cm PF = Physical Fitness \hskip1cm FS = Financial Savings \hskip1cm WL = Work-Life Balance
%%%%%
%{\rowcolors{1}{white!80!black!20}{white!70!black!60}
%\begin{center}
%\begin{tabular}{ | m{10em} | m{5cm}| m{5cm} | } 
%\hline
 %& Positively Impacts & Negatively Impacts \\ 
%\hline
%Personal Motor Vehicle & \QO SL\hskip0.5cm \QO PF \hskip0.5cm \QO FS \hskip0.5cm \QO WL 
 %    & \QO SL\hskip0.5cm \QO PF \hskip0.5cm \QO FS \hskip0.5cm \QO WL \\ 
%\hline
%Personal Bike & \QO SL\hskip0.5cm \QO PF \hskip0.5cm \QO FS \hskip0.5cm \QO WL 
%& \QO SL\hskip0.5cm \QO PF \hskip0.5cm \QO FS \hskip0.5cm \QO WL \\ 
%\hline
%Bike Sharing & \QO SL\hskip0.5cm \QO PF \hskip0.5cm \QO FS \hskip0.5cm \QO WL 
%& \QO SL\hskip0.5cm \QO PF \hskip0.5cm \QO FS \hskip0.5cm \QO WL \\ 
%\hline
%Ride-Sharing & \QO SL\hskip0.5cm \QO PF \hskip0.5cm \QO FS \hskip0.5cm \QO WL 
%& \QO SL\hskip0.5cm \QO PF \hskip0.5cm \QO FS \hskip0.5cm \QO WL \\ 
%\hline
%Taxi & \QO SL\hskip0.5cm \QO PF \hskip0.5cm \QO FS \hskip0.5cm \QO WL 
%& \QO SL\hskip0.5cm \QO PF \hskip0.5cm \QO FS \hskip0.5cm \QO WL \\ 
%\hline
%Public Bus& \QO SL\hskip0.5cm \QO PF \hskip0.5cm \QO FS \hskip0.5cm \QO WL 
%& \QO SL\hskip0.5cm \QO PF \hskip0.5cm \QO FS \hskip0.5cm \QO WL \\ 
%\hline
%Cable Aereo & \QO SL\hskip0.5cm \QO PF \hskip0.5cm \QO FS \hskip0.5cm \QO WL 
%& \QO SL\hskip0.5cm \QO PF \hskip0.5cm \QO FS \hskip0.5cm \QO WL \\ 
%\hline
%Other: \Qline{2cm} & \QO SL\hskip0.5cm \QO PF \hskip0.5cm \QO FS \hskip0.5cm \QO WL 
%& \QO SL\hskip0.5cm \QO PF \hskip0.5cm \QO FS \hskip0.5cm \QO WL \\ 
%\hline
%\end{tabular}
%\end{center}

\subsection*{Potential Smart Mobility Initiatives in Manizales, Colombia}

\Qitem{\Qq{Did you know that CityBioBike is free to use?}\hskip1cm\QO{} yes \hskip1cm \QO{} no}{}

If the below infrastructure or initiatives were implemented in Manizales, Colombia, how likely are you to start using CityBioBike? \\

\QItem[a]{ \Qq{Smart Phone Application} \Qtab{7.5cm}{very unlikely \Qrating{5} very likely}}

\QItem[b]{ \Qq{Complete streets} \Qtab{7.5cm}{very unlikely \Qrating{5} very likely}}\\
Complete streets are streets that separate motorvehicles from bicyles and from pedestrians by creating separate lanes for each mode of transportation.

\QItem[c]{ \Qq{Protected bike-only lane} \Qtab{7.5cm}{very unlikely \Qrating{5} very likely}}

\QItem[d]{ \Qq{Electric-assisted bike} \Qtab{7.5cm}{very unlikely \Qrating{5} very likely}}

\QItem[e]{ \Qq{Dockless bike system} \Qtab{7.5cm}{very unlikely \Qrating{5} very likely}}

\Qitem{{For the questions between 11a. and 11e. that you answered very unlikely or unlikely, please select why below.}
    \begin{Qlist}
    \item[]\QO{} Not necessary \hskip2.5cm \QO{} Doesn't apply to me \hskip2cm \QO{} I'm not good with technology 
    \item[]\QO{} Not safe \hskip3.3cm \QO{} Not practical  \hskip3.2cm \QO{} Other: \Qline{3cm}
    \end{Qlist}
}
\Qitem{{For the questions between 14a. and 14e. that you answered very likely or likely, please select why below.}
    \begin{Qlist}
    \item[]\QO{} Improves efficiency in commute \hskip0.5cm \QO{} Decreases stress \hskip2.5cm \QO{} Envrionmentally friendly
    \item[]\QO{}  Improves my personal safety \hskip1cm \QO{} Increases my physical fitness 
    \item[]\QO{} Other: \Qline{6cm}
    \end{Qlist}
}

\Qitem{Do you have any other comments about the current or potential transportation and it's impact on your quality of life? Write them in the space below.}

\end{document}