\documentclass[a4paper,10pt,BCOR10mm,oneside,headsepline]{scrartcl}
\usepackage[ngerman]{babel}
\usepackage[utf8]{inputenc}
\usepackage{wasysym}% provides \ocircle and \Box
\usepackage{enumitem}% easy control of topsep and leftmargin for lists
\usepackage{color}% used for background color
\usepackage{forloop}% used for \Qrating and \Qlines
\usepackage{ifthen}% used for \Qitem and \QItem
\usepackage{typearea}
\usepackage{tabularcalc}
\usepackage{geometry}
\usepackage[table]{xcolor}
\areaset{17cm}{26cm}
\usepackage{scrpage2}

%%%%%%%%%%%%%%%%%%%%%%%%%%%%%%%%%%%%%%%%%%%%%%%%%%%%%%%%%%%%
%% Beginning of questionnaire command definitions %%
%%%%%%%%%%%%%%%%%%%%%%%%%%%%%%%%%%%%%%%%%%%%%%%%%%%%%%%%%%%%
%%
%% 2010, 2012 by Sven Hartenstein
%% mail@svenhartenstein.de
%% http://www.svenhartenstein.de
%%
%% Please be warned that this is NOT a full-featured framework for
%% creating (all sorts of) questionnaires. Rather, it is a small
%% collection of LaTeX commands that I found useful when creating a
%% questionnaire. Feel free to copy and adjust any parts you like.
%% Most probably, you will want to change the commands, so that they
%% fit your taste.
%%
%% Also note that I am not a LaTeX expert! Things can very likely be
%% done much more elegant than I was able to. If you have suggestions
%% about what can be improved please send me an email. I intend to
%% add good tipps to my website and to name contributers of course.
%%
%% 10/2012: Thanks to karathan for the suggestion to put \noindent
%% before \rule!

%% \Qq = Questionaire question. Oh, this is just too simple. It helps
%% making it easy to globally change the appearance of questions.
\newcommand{\Qq}[1]{\textbf{#1}}

%% \QO = Circle or box to be ticked. Used both by direct call and by
%% \Qrating and \Qlist.
\newcommand{\QO}{$\Box$}% or: $\ocircle$

%% \Qrating = Automatically create a rating scale with NUM steps, like
%% this: 0--0--0--0--0.
\newcounter{qr}
\newcommand{\Qrating}[1]{\QO\forloop{qr}{1}{\value{qr} < #1}{---\QO}}

%% \Qline = Again, this is very simple. It helps setting the line
%% thickness globally. Used both by direct call and by \Qlines.
\newcommand{\Qline}[1]{\noindent\rule{#1}{0.6pt}}

%% \Qlines = Insert NUM lines with width=\linewith. You can change the
%% \vskip value to adjust the spacing.
\newcounter{ql}
\newcommand{\Qlines}[1]{\forloop{ql}{0}{\value{ql}<#1}{\vskip0em\Qline{\linewidth}}}

%% \Qlist = This is an environment very similar to itemize but with
%% \QO in front of each list item. Useful for classical multiple
%% choice. Change leftmargin and topsep accourding to your taste.
\newenvironment{Qlist}{%
\renewcommand{\labelitemi}{\QO}
\begin{itemize}[leftmargin=1.5em,topsep=-.5em]
}{%
\end{itemize}
}

%% \Qtab = A "tabulator simulation". The first argument is the
%% distance from the left margin. The second argument is content which
%% is indented within the current row.
\newlength{\qt}
\newcommand{\Qtab}[2]{
\setlength{\qt}{\linewidth}
\addtolength{\qt}{-#1}
\hfill\parbox[t]{\qt}{\raggedright #2}
}

%% \Qitem = Item with automatic numbering. The first optional argument
%% can be used to create sub-items like 2a, 2b, 2c, ... The item
%% number is increased if the first argument is omitted or equals 'a'.
%% You will have to adjust this if you prefer a different numbering
%% scheme. Adjust topsep and leftmargin as needed.
\newcounter{itemnummer}
\newcommand{\Qitem}[2][]{% #1 optional, #2 notwendig
\ifthenelse{\equal{#1}{}}{\stepcounter{itemnummer}}{}
\ifthenelse{\equal{#1}{a}}{\stepcounter{itemnummer}}{}
\begin{enumerate}[topsep=2pt,leftmargin=2.8em]
\item[\textbf{\arabic{itemnummer}#1.}] #2
\end{enumerate}
}

%% \QItem = Like \Qitem but with alternating background color. This
%% might be error prone as I hard-coded some lengths (-5.25pt and
%% -3pt)! I do not yet understand why I need them.
\definecolor{bgodd}{rgb}{0.8,0.8,0.8}
\definecolor{bgeven}{rgb}{0.9,0.9,0.9}
\newcounter{itemoddeven}
\newlength{\gb}
\newcommand{\QItem}[2][]{% #1 optional, #2 notwendig
\setlength{\gb}{\linewidth}
\addtolength{\gb}{-5.25pt}
\ifthenelse{\equal{\value{itemoddeven}}{0}}{%
\noindent\colorbox{bgeven}{\hskip-3pt\begin{minipage}{\gb}\Qitem[#1]{#2}\end{minipage}}%
\stepcounter{itemoddeven}%
}{%
\noindent\colorbox{bgodd}{\hskip-3pt\begin{minipage}{\gb}\Qitem[#1]{#2}\end{minipage}}%
\setcounter{itemoddeven}{0}%
}
}

%%%%%%%%%%%%%%%%%%%%%%%%%%%%%%%%%%%%%%%%%%%%%%%%%%%%%%%%%%%%
%% End of questionnaire command definitions %%
%%%%%%%%%%%%%%%%%%%%%%%%%%%%%%%%%%%%%%%%%%%%%%%%%%%%%%%%%%%%

\begin{document}
\renewcommand{\QO}{$\ocircle$}% Use circles now instead of boxes.
%%\begin{center}
%%\textbf{\huge Questionnaire \LaTeX}
%%\end{center}\vskip1em

\noindent Thank you for participating in this study; your honesty and responses are appreciated.
Your responses to the questions below will remain annonynous and confidential.
You do not have to answer any question that you do not want to. This study is not
associated with any public or private organizations. Analysis of data will be for
academic use only at the University of Pittsburgh (EE. UU.). 

%\noindent \textit{Please note that no tabular environment is used in
%this example questionnaire. Of course, you could use tabular to
%create more complex layout.}

\subsection*{About you}

\Qitem{ \Qq{Please select your age range: }\hskip0.4cm \QO{}
18-28 \hskip0.4cm \QO{} 29-40 \hskip0.4cm \QO{} 41-50 \hskip0.4cm \QO{} 51-60 \hskip0.4cm \QO{} 61+}

\vspace{0.1cm}

\Qitem{ \Qq{Please select the gender that you identify with: }\hskip0.4cm \QO{}
Female \hskip0.4cm \QO{} Male \hskip0.4cm Other: \Qline{2.5cm} }

\vspace{0.1cm}

\Qitem{\Qq{When it comes transportation, which is the most important factor to you?}
 \begin{Qlist} 
    \item[]\QO{} Safety \hskip0.5cm \QO{} Travel time \hskip0.5cm \QO{} Cost \hskip0.5cm \QO{} Comfortability \hskip0.5cm \QO{} Other: \Qline{3cm}
 \end{Qlist}
}

\Qitem{ \Qq{Have you ever heard of the bike sharing system called CityBioBike?}
\QO{} yes \hskip0.2cm \QO{} no \hskip0.2cm \QO{} not sure}{}

\vspace{0.1cm}

\Qitem{\Qq{Do you live within a 10 minute walk from a CityBioBike station?}
\QO{} yes \hskip0.5cm \QO{} no \hskip0.5cm \QO{} not sure}{}

\vspace{0.1cm}

\Qitem{ \Qq{Have you ever used CityBioBike? (If no, go to question 14.)}
\hskip1.5cm\QO{} yes \hskip1cm \QO{} no}{}
\Qitem{\Qq{Are you a frequent user? (If no, go to question 10.)}
\hskip1cm\QO{} yes \hskip1cm \QO{} no}{}
\renewcommand{\QO}{$\ocircle$}
\Qitem{ \Qq{How many times do you use CityBioBike on avg in a week?: }\hskip0.2cm \QO{}
0-4 \hskip0.2cm \QO{} 5-8 \hskip0.2cm \QO{} 9-12 \hskip0.2cm \QO{} 13+}
\vspace{0.1cm}
\renewcommand{\QO}{$\Box$}
\Qitem{\Qq{What are the main reasons you use CityBioBike? Select all that apply.} 
    \begin{Qlist}
    \item[]\QO{} To grocery shop \hskip0.5cm \QO{} To go to work \hskip0.5cm \QO{} To exercise \hskip0.5cm \QO{} To save money \hskip0.5cm \QO{} To go to school
    \item[]\QO{} To visit friends and family \hskip0.5cm \QO{} To enjoy the nature and mountains \hskip0.5cm \QO{} Other: \Qline{3cm}
    \end{Qlist}
}
\vspace{0.1cm}
\renewcommand{\QO}{$\ocircle$}

\subsection*{Current Bicycle Initiatives in Manizales, Colombia}

\vspace{0.1cm}

\hskip0.4cm \textbf{10. Please rate how CityBioBike is doing with respect to each category listed below.} \\
\\
\QItem[a]{ \Qq{Cleanliness} \hskip5.7cm {not clean\Qrating{5} very clean}}

\QItem[c]{ \Qq{Quality} \hskip5.9cm {breaks a lot \Qrating{5} rarely breaks}}

\QItem[e]{ \Qq{Safety}\hskip6.9cm {not safe \Qrating{5} very safe}}

\QItem[b]{ \Qq{Registration Process} \hskip3cm {too complicated \Qrating{5} very easy}}

\QItem[f]{ \Qq{Renting Process}\hskip3.9cm {too complicated \Qrating{5} very easy}}

\QItem[d]{ \Qq{Bike availability}\hskip4.1cm {never available \Qrating{5} always available}}

\QItem[f]{ \Qq{Bike Station Locations}\hskip2cm {needs more locations \Qrating{5} perfect as is}}\\

\vspace{0.1cm}
\Qitem{Please provide any suggestions about how CityBioBike can improve in any of the areas listed above. } \Qline{17cm}\\ \Qline{17cm}\\ \Qline{17cm}\\

\hskip0cm \textbf{12. Please rate how the city of Manizales is doing with respect to each category listed below. }\\

\QItem[a]{ \Qq{Bike racks} \hskip5.7cm {need more \Qrating{5} perfect as is}}

\QItem[c]{ \Qq{Bike only lanes} \hskip4.8cm {need more \Qrating{5} perfect as is}}

\QItem[e]{ \Qq{Road Signs for bikers}\hskip3.8cm {need more \Qrating{5} perfect as is}}

\QItem[b]{ \Qq{Bike safety education \& laws} \hskip2.4cm {need more \Qrating{5} perfect as is}}\\

\vspace{0.1cm}
\Qitem{Please provide any suggestions about how Manizales can improve in any of the areas listed above. } \Qline{17cm}\\ \Qline{17cm}\\ \\
\textbf{Proceed to question 16.}

\subsection*{Potential Bicycle Initiatives in Manizales, Colombia}
\vspace{0.1cm}

\renewcommand{\QO}{$\ocircle$}
\Qitem{\Qq{Did you know that CityBioBike is free to use?}\hskip1cm\QO{} yes \hskip1cm \QO{} no}{}
\vspace{0.1cm}
\renewcommand{\QO}{$\Box$}
\Qitem{\Qq{What are the main reasons that you do not use the CityBioBike bike sharing system? Select all that apply.}
    \begin{Qlist}
    \item[]\QO{} I prefer the bus \hskip0.3cm \QO{} I own a motorvehicle \hskip0.3cm \QO{} No bike stations near me \hskip0.5cm \QO{} I don't think it's safe
    \item[]\QO{} My destination is within walking distance \hskip0.5cm \QO{} Too many hills on my route \hskip0.3cm \QO{} Other: \Qline{2cm}
    \end{Qlist}
}
\vspace{0.1cm}

\renewcommand{\QO}{$\ocircle$}
16. How likely are you to start using or continue using CityBioBike if the following were offered? \\

\QItem[a]{ \Qq{Bike availability phone app} \Qtab{7.5cm}{very unlikely \Qrating{5} very likely}}

\QItem[c]{ \Qq{Protected bike-only lane} \Qtab{7.5cm}{very unlikely \Qrating{5} very likely}}

\QItem[e]{ \Qq{More CityBioBike rental stations} \Qtab{7.5cm}{very unlikely \Qrating{5} very likely}}

\QItem[e]{ \Qq{More CityBioBike bikes circulating} \Qtab{7.5cm}{very unlikely \Qrating{5} very likely}}

\QItem[d]{ \Qq{Electric-assisted bike} \Qtab{7.5cm}{very unlikely \Qrating{5} very likely}}

\QItem[e]{ \Qq{Stationless bike sharing system} \Qtab{7.5cm}{very unlikely \Qrating{5} very likely}}

\QItem[e]{ \Qq{Free bike helmet} \Qtab{7.5cm}{very unlikely \Qrating{5} very likely}}\\
\vspace{0.1cm}
\Qitem{Do you have suggestions to help improve the bike sharing system in Manizales, Colombia?} \Qline{17cm}\\ \Qline{17cm}\\ \Qline{17cm}\\

\end{document}