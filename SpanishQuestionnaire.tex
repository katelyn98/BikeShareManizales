\documentclass[a4paper,11pt,BCOR10mm,oneside,headsepline]{scrartcl}
\usepackage[ngerman]{babel}
\usepackage[utf8]{inputenc}
\usepackage{wasysym}% provides \ocircle and \Box
\usepackage{enumitem}% easy control of topsep and leftmargin for lists
\usepackage{color}% used for background color
\usepackage{forloop}% used for \Qrating and \Qlines
\usepackage{ifthen}% used for \Qitem and \QItem
\usepackage{typearea}
\usepackage{tabularcalc}
\usepackage{geometry}
\usepackage[table]{xcolor}
\areaset{17cm}{26cm}
\usepackage{scrpage2}

%%%%%%%%%%%%%%%%%%%%%%%%%%%%%%%%%%%%%%%%%%%%%%%%%%%%%%%%%%%%
%% Beginning of questionnaire command definitions %%
%%%%%%%%%%%%%%%%%%%%%%%%%%%%%%%%%%%%%%%%%%%%%%%%%%%%%%%%%%%%
%%
%% 2010, 2012 by Sven Hartenstein
%% mail@svenhartenstein.de
%% http://www.svenhartenstein.de
%%
%% Please be warned that this is NOT a full-featured framework for
%% creating (all sorts of) questionnaires. Rather, it is a small
%% collection of LaTeX commands that I found useful when creating a
%% questionnaire. Feel free to copy and adjust any parts you like.
%% Most probably, you will want to change the commands, so that they
%% fit your taste.
%%
%% Also note that I am not a LaTeX expert! Things can very likely be
%% done much more elegant than I was able to. If you have suggestions
%% about what can be improved please send me an email. I intend to
%% add good tipps to my website and to name contributers of course.
%%
%% 10/2012: Thanks to karathan for the suggestion to put \noindent
%% before \rule!

%% \Qq = Questionaire question. Oh, this is just too simple. It helps
%% making it easy to globally change the appearance of questions.
\newcommand{\Qq}[1]{\textbf{#1}}

%% \QO = Circle or box to be ticked. Used both by direct call and by
%% \Qrating and \Qlist.
\newcommand{\QO}{$\Box$}% or: $\ocircle$

%% \Qrating = Automatically create a rating scale with NUM steps, like
%% this: 0--0--0--0--0.
\newcounter{qr}
\newcommand{\Qrating}[1]{\QO\forloop{qr}{1}{\value{qr} < #1}{---\QO}}

%% \Qline = Again, this is very simple. It helps setting the line
%% thickness globally. Used both by direct call and by \Qlines.
\newcommand{\Qline}[1]{\noindent\rule{#1}{0.6pt}}

%% \Qlines = Insert NUM lines with width=\linewith. You can change the
%% \vskip value to adjust the spacing.
\newcounter{ql}
\newcommand{\Qlines}[1]{\forloop{ql}{0}{\value{ql}<#1}{\vskip0em\Qline{\linewidth}}}

%% \Qlist = This is an environment very similar to itemize but with
%% \QO in front of each list item. Useful for classical multiple
%% choice. Change leftmargin and topsep accourding to your taste.
\newenvironment{Qlist}{%
\renewcommand{\labelitemi}{\QO}
\begin{itemize}[leftmargin=1.5em,topsep=-.5em]
}{%
\end{itemize}
}

%% \Qtab = A "tabulator simulation". The first argument is the
%% distance from the left margin. The second argument is content which
%% is indented within the current row.
\newlength{\qt}
\newcommand{\Qtab}[2]{
\setlength{\qt}{\linewidth}
\addtolength{\qt}{-#1}
\hfill\parbox[t]{\qt}{\raggedright #2}
}

%% \Qitem = Item with automatic numbering. The first optional argument
%% can be used to create sub-items like 2a, 2b, 2c, ... The item
%% number is increased if the first argument is omitted or equals 'a'.
%% You will have to adjust this if you prefer a different numbering
%% scheme. Adjust topsep and leftmargin as needed.
\newcounter{itemnummer}
\newcommand{\Qitem}[2][]{% #1 optional, #2 notwendig
\ifthenelse{\equal{#1}{}}{\stepcounter{itemnummer}}{}
\ifthenelse{\equal{#1}{a}}{\stepcounter{itemnummer}}{}
\begin{enumerate}[topsep=2pt,leftmargin=2.8em]
\item[\textbf{\arabic{itemnummer}#1.}] #2
\end{enumerate}
}

%% \QItem = Like \Qitem but with alternating background color. This
%% might be error prone as I hard-coded some lengths (-5.25pt and
%% -3pt)! I do not yet understand why I need them.
\definecolor{bgodd}{rgb}{0.8,0.8,0.8}
\definecolor{bgeven}{rgb}{0.9,0.9,0.9}
\newcounter{itemoddeven}
\newlength{\gb}
\newcommand{\QItem}[2][]{% #1 optional, #2 notwendig
\setlength{\gb}{\linewidth}
\addtolength{\gb}{-5.25pt}
\ifthenelse{\equal{\value{itemoddeven}}{0}}{%
\noindent\colorbox{bgeven}{\hskip-3pt\begin{minipage}{\gb}\Qitem[#1]{#2}\end{minipage}}%
\stepcounter{itemoddeven}%
}{%
\noindent\colorbox{bgodd}{\hskip-3pt\begin{minipage}{\gb}\Qitem[#1]{#2}\end{minipage}}%
\setcounter{itemoddeven}{0}%
}
}

%%%%%%%%%%%%%%%%%%%%%%%%%%%%%%%%%%%%%%%%%%%%%%%%%%%%%%%%%%%%
%% End of questionnaire command definitions %%
%%%%%%%%%%%%%%%%%%%%%%%%%%%%%%%%%%%%%%%%%%%%%%%%%%%%%%%%%%%%

\begin{document}
\renewcommand{\QO}{$\ocircle$}% Use circles now instead of boxes.
%%\begin{center}
%%\textbf{\huge Questionnaire \LaTeX}
%%\end{center}\vskip1em

\noindent Gracias por participar en esta investigación; se aprecio sus respuestas 
y su honradez. Sus respuestas serán anónimas y confidenciales. No se tiene que 
responder ningunas preguntas que no quiere. Esta investigación no está afiliada a 
ninguna organización publica o privada. El análisis de los data de esta investigación 
sólo utilizará para académica en la Universidad de Pittsburgh (EE. UU.).  

%\noindent \textit{Please note that no tabular environment is used in
%this example questionnaire. Of course, you could use tabular to
%create more complex layout.}

\subsection*{Sobre Usted}

\Qitem{ \Qq{¿Cuántos años tiene?: }\hskip0.4cm \QO{}
18 - 28 \hskip0.4cm \QO{} 29 - 40 \hskip0.4cm \QO{} 41 - 50 \hskip0.4cm \QO{} 51 - 60 \hskip0.4cm \QO{} 61+}

\vspace{0.1cm}

\Qitem{ \Qq{¿Con cuál género se identifica?: }\hskip0.4cm \QO{}
Hombre \hskip0.4cm \QO{} Mujer \hskip0.4cm Otro: \Qline{2.5cm} }
\vspace{0.1cm}

\Qitem{\Qq{¿En relación con transporte, cuál factor es el más important para usted?}
 \begin{Qlist} 
    \item[]\QO{} la seguridad \hskip0.5cm \QO{} tiempo de viaje \hskip0.5cm \QO{} el costa \hskip0.5cm \QO{} Comodidad \hskip0.5cm \QO{} a pedido \hskip0.5cm 
 \end{Qlist}
}
\vspace{0.1cm}

\subsection*{Las Iniciativas de Bicis en Manizales, Colombia}

\Qitem{ \Qq{¿Supo que hay un sistema de bicicleta pública en Manizales?}\\
\QO{} Sí \hskip2cm \QO{} No \hskip2cm \QO{} No sé}{}

\vspace{0.1cm}

\Qitem{\Qq{¿Usted vive a menos de a 10 minutos a pie de una estación de bicicleta pública?}\\
\QO{} Sí \hskip2cm \QO{} No \hskip2cm \QO{} No sé}{}

\vspace{0.1cm}

\Qitem{ \Qq{¿Alguna vez ha usado el sistema de bicicleta pública en Manizales? (Si no, vaya a la pregunta 13)}
\hskip0.5cm\QO{} Sí \hskip1cm \QO{} No}{}
\vspace{0.1cm}
\Qitem{\Qq{¿Lo usa con frecuencia? (Si no, vaya a la pregunta 13)}
\hskip1cm\QO{} Sí \hskip1cm \QO{} No}{}
\vspace{0.1cm}
\renewcommand{\QO}{$\ocircle$}
\Qitem{ \Qq{¿En promedio, cuántas veces usa el sistema de bicicleta pública en una semana?: }\\\hskip2cm \QO{}
0 - 4 \hskip2cm \QO{} 5 - 8 \hskip2cm \QO{} 9 - 12 \hskip2cm \QO{} 13+}
\vspace{0.1cm}
\renewcommand{\QO}{$\Box$}
\Qitem{\Qq{¿Cuáles son sus razones principales para usando el sistema de bicicleta pública? Marque todo lo que aplique.} 
}
\begin{Qlist}
    \item[]\QO{} Ir al supermercado \hskip0.3cm \QO{} Ir al trabajo \hskip0.3cm \QO{} Hacer ejercicio \hskip0.3cm \QO{} Ahorrar dinero  \hskip0.3cm \QO{} Ir a la escuela
    \item[]\QO{} Visitar amigos y parientes \hskip0.5cm \QO{} Disfrutar natura \hskip0.5cm \QO{} Otro: \Qline{3cm}
    \end{Qlist}
\vspace{0.3cm}
\renewcommand{\QO}{$\ocircle$}

\vspace{0.1cm}

\textbf{10. Por favor usted clasifica cómo el sistema de bicicleta pública en Manizales es en relación con cada factor enumerados a continuación.} \\
\\
\QItem[a]{ \Qq{La limpieza} \Qtab{6.5cm} {no limpio \Qrating{5} muy limpio}}

\QItem[b]{ \Qq{La cualidad} \Qtab{6.5cm} {rompe mucho \Qrating{5} nunca rompe}}

\QItem[c]{ \Qq{Seguridad}\Qtab{6.5cm}{no es seguro \Qrating{5} es muy seguro}}

\QItem[d]{ \Qq{Proceso de registro} \hskip2.7cm {demasiado complicado \Qrating{5} muy fácil}}

\QItem[e]{ \Qq{Proceso de rentar}\hskip3.1cm {demasiado complicado\Qrating{5} muy fácil}}

\QItem[f]{ \Qq{Disponibilidad de bicis}\Qtab{5cm} {nunca disponibilidad \Qrating{5} siempre disponibilidad}}

\QItem[g]{ \Qq{Ubicaciónes de las estaciones}\hskip0.5cm {necesita más ubicaciónes \Qrating{5} lo es perfecto}}

\QItem[h]{ \Qq{Aparcamiento para bicis}\hskip1.5cm {necesita más espacios \Qrating{5} lo es perfecto}}

\QItem[i]{ \Qq{Carriles sólo para bicis}\hskip3.0cm {necesita más \Qrating{5} lo es perfecto}}
\\
\\
\vspace{0.1cm}
\Qitem{\Qq{Por favor provide cualquier sugerencia sobre cómo Manizales puede mejorar en cualquier factor enumerados arriba.}} \Qline{17cm}\\ \Qline{17cm}\\ \Qline{17cm}

\textbf{Continúa a la pregunta 15}\\

\vspace{0.1cm}
\renewcommand{\QO}{$\Box$}
\Qitem{\Qq{Por favor usted clasifica cuáles factores que desempeña un papel en por qué usted no usa el sistema de bicis con frecuencia.}
    \begin{Qlist}
    \item[]\QO{} La seguridad en la calle \hskip1cm \QO{} La cualidad de bicis \hskip1cm \QO{} No estaciones cerca de mi
    \item[]\QO{} Tiempo de viaje demasiado largo \hskip1cm \QO{} No es fiable \hskip1cm \QO{} Prefiero el bús  
    \item[]\QO{} Demasiado muchas colinas en mi ruta \hskip1cm \QO{} Puedo caminar a mi destinación 
    \item[]\QO{} Otro: \Qline{2cm}\\
    \end{Qlist}
}

\subsection*{Potencial Iniciativas de Bicis en Manizales, Colombia}
\vspace{0.1cm}

\renewcommand{\QO}{$\ocircle$}
\Qitem{\Qq{¿Supo que el sistema de bicicleta pública es gratis por usar?}\hskip1cm\QO{} Sí \hskip1cm \QO{} No}{}
\vspace{0.1cm}
\renewcommand{\QO}{$\Box$}
\Qitem{\Qq{¿Cuáles son sus razones principales para no usando el sistema de bicicleta pública? Marque todo lo que aplique.}
    \begin{Qlist}
    \item[]\QO{} La seguridad en la calle \hskip1cm \QO{} La cualidad de bicis \hskip1cm \QO{} No estaciones cerca de mi
    \item[]\QO{} Tiempo de viaje demasiado largo \hskip1cm \QO{} No es fiable \hskip1cm \QO{} Prefiero el bús  
    \item[]\QO{} Demasiado muchas colinas en mi ruta \hskip1cm \QO{} Puedo caminar a mi destinación 
    \item[]\QO{} Otro: \Qline{2cm}\\
    \end{Qlist}

}
\vspace{0.1cm}

\renewcommand{\QO}{$\ocircle$}
\textbf{15. ¿Cómo probable lo es que empezará o continuará usando el sistema de bicicleta pública si los factors enumerados a continuación será ofrecer?} \\

\QItem[a]{ \Qq{Phone App para disponibilidad de bicis} \Qtab{7.5cm}{imporbable \Qrating{3} probable}}

\QItem[c]{ \Qq{Carril protegido para bici} \Qtab{7.5cm}{imporbable \Qrating{3} probable}}

\QItem[e]{ \Qq{Más estaciones de bicis} \Qtab{7.5cm}{imporbable \Qrating{3} probable}}

\QItem[e]{ \Qq{Más estaciones cerca de la parada de bús} \Qtab{7.5cm}{imporbable \Qrating{3} probable}}

\QItem[e]{ \Qq{Más bicis electricas cerca de mi} \Qtab{7.5cm}{imporbable \Qrating{3} probable}}

\QItem[e]{ \Qq{Una sistema sin estaciones} \Qtab{7.5cm}{imporbable \Qrating{3} probable}}
\\
\vspace{0.1cm}

\Qitem{\Qq{Por favor provide cualquier sugerencia sobre cómo Manizales puede mejorar el sistema de bicicleta pública.}} \Qline{17cm}\\ \Qline{17cm}\\ \Qline{17cm}\\


\textbf{Gracias por responder a las preguntas y gracias por tu tiempo! Tiene un día genial!}

\end{document}