\documentclass[12pt]{article}
\usepackage[letterpaper]{geometry}
\usepackage{times}
\usepackage[utf8]{inputenc}
\usepackage[english]{babel}
\geometry{top=1.0in, bottom=1.0in, left=1.0in, right=1.0in}
\usepackage{fancyhdr}
\usepackage{csquotes}
\usepackage{hyperref}

%To make sure we actually have header 0.5in away from top edge
%12pt is one-sixth of an inch. Subtract this from 0.5in to get headsep value
\setlength\headsep{0.333in}
\usepackage{setspace}
\usepackage[style=apa]{biblatex}
\addbibresource{references.bib}
\usepackage{indentfirst}

\doublespacing
\begin{document}
\input{title}

\subsection*{Introduction}
Cities are getting bigger and computers are getting faster, and yet the skies are full of air pollution. But, economies are booming and the collaboration between citizens is resulting in life-changing technology that fosters on-demand, instant gratification characteristics. Such technology can prolong life and destroy boredom, but it can also be used to mitigate the impacts of climate change and augment outdated systems, such as a city's transportation system. The design of such a system, from its infrastructure to its user interactions, is imperative for a society to thrive. In hopes to amplify any given system in a city, technology and infrastructure are meticulously coalesced resulting in what researchers hope will produce a safer, reliable, and sustainable system. This combination of technology and infrastructure is highlighted in the definition of a smart city (\cite{montoya2017acceso}). To clarify, by definition a city only needs to have one of the components that make up a smart city to be labeled as a smart city (\cite{montoya2017acceso}). 

 Transportation within a city is vital and significantly impacts the economy and quality of life (\cite{dávalos_maldonado_polit_2016}; \cite{zuluaga_garcía_2017}). A city and its metropolitan area, or outskirts, need to have a strong transportation system to successfully and reliably transport copious amounts of workers to and from "downtown" everyday; this means a transportation system needs to be designed to manage anywhere between less than 100,000 people to 1 million people or more everyday. Historically, cities are known for large highways with up to twelve lanes and very dense, complex road networks to mitigate traffic jams. However, Braess's Paradox validates how the creation of these dense road networks are actually making the traffic worse (\cite{wikipedia_2020}). In fact, Deliotte reported in their \textit{Smart Mobility Research Report} that, "The average American spends about 34 hours every year sitting in traffic" (pg. 2 \cite{DeliotteReport}). Clearly, the traditional cityscape fosters a system built to prioritize personal motor vehicles. The presence of cars in cities has lead to a crucial decline in sustainability including an increase in carbon emission, an increase in traffic jams, and an inefficient use of land which all negatively impact a citizen's quality of life (\cite{dávalos_maldonado_polit_2016}). Only recently have these inefficiencies been taken seriously. Consequently, urban planners, researchers, and entrepreneurs are rapidly prototyping new transportation systems in hopes to foster a smart transportation system while addressing issues related to traffic congestion, efficiency, and environmental impact. One smart transportation that is rapidly growing popularity in cities across the world is the bike sharing systems. Evidently, this combination of technologies and transportation systems is enhancing the effectiveness and efficiency of operations in cities and increasing the overall quality of life.  (\cite{montoya2017acceso}, \cite{dávalos_maldonado_polit_2016}).

Alternative approaches to the traditional transportation system aim to, "...reduce congestion and foster faster, greener, and cheaper transportation options" while enhancing the quality of life of the community (\cite{DeliotteReport}). These alternative approaches are multifaceted from the design of streets to the types of modes of transportation offered which includes ride sharing systems, bicycle commuting, car sharing, and on-demand ride services (pg. 4 \cite{DeliotteReport}). Three of those four presented alternative modes of transportation to traditional transportation involves cars. Only one alternative approach aims to improve the transportation system without the use of carbon emitting vehicles: bicycle commuting. 

The main objective is not to totally eliminate motor vehicles. In fact, that would be taking our society back to pre-automobile times...which also is not good for efficiency or an economy. To successfully transform urban mobility in a city to a multi-modal, sustainable system will take time. One of the goals of this process includes converting citizens who \textit{could} bike to work to do so instead of driving a car - the fewer number of cars on the road results in less congestion, less noise/air pollution, and safer roads for more active modes of transit like bicycling. Asking everyone to own a bicycle seems far-fetched; so, in 1995 a group of urban mobility enthusiasts in Amsterdam created the first public bike sharing system (\cite{zuluaga_garcía_2017}). Since then, bicycle commuting and bike sharing programs have been popping up around the world, with a high volume in Europe and cities that tend to have overall flat terrain (\cite{DeliotteReport}). Some cities however are struggling with integrating the system into their current transportation system, especially cities in developing countries and cities with rugged terrain featuring steep hills and bluffs (\cite{zuluaga_garcía_2017}). As a matter of fact, developing countries such as Brazil, Chile, and Mexico, didn't start operating a public bike sharing system until after 2008 (\cite{midgley_2011}). Other countries in South America, such as Colombia, have been following the footsteps of Brazil and Chile. Several cities within Colombia have seen the public bike sharing system model by now, but some cities have experienced more success than others. This disparity in success provides a motive to investigate the public opinions of the bike sharing system in Manizales, Colombia that is not experiencing the overwhelming success typically brought by bike sharing systems. 
\subsubsection*{Statement of Purpose}
The objective behind this study is to better understand how adults in Manizales, Colombia view the current bike sharing system called \textit{Manizales En Bici}. This study attempts to determine which certain factors such as safety, efficiency, and reliability, the people of Manizales consider when assessing the quality of the current bike sharing system. Additionally, this study seeks to identify how individuals predict possible improvements relating to infrastructure, technology, and safety to the current bike sharing system might impact their use and view of this bike sharing system in the future.
\subsubsection*{Research Question}
Which certain factors such as safety, efficiency, and reliability influence how adults view and assess the current bike sharing system, \textit{Manizales En Bici}, in Manizales, Colombia? How do potential improvements relating to infrastructure, technology, and safety to the current bike sharing system, \textit{Manizales En Bici}, impact their use and view of the system in the future?

\subsubsection*{Sub-problems}
\begin{itemize}
    \item How do current users of \textit{Manizales En Bici} view the safety, efficiency, and reliability of its' infrastructure and initiatives?
    \item To what extent does the safety of the \textit{Manizales En Bici} infrastructure and initiatives influence how an individual views and assesses it?
    \item To what extent does the efficiency of the \textit{Manizales En Bici} infrastructure influence how an individual views and assess it?
    \item To what extent does the reliability of the \textit{Manizales En Bici} infrastructure influence how an individual views and assess it?
    \item What factors are most valued when an individual evaluates the potential benefit of future improvements to \textit{Manizales En Bici} bike sharing system?
\end{itemize}
\subsubsection*{Hypothesis}
Given the nature of the landscape in Manizales, Colombia, efficiency and reliability will be the most significant contributing factors that influence how adults assess and evaluate the \textit{Manizales En Bici} bike sharing system. The terrain in Manizales, Colombia is very mountainous with some streets featuring inclinations of greater than 18\% grade (\cite{cardona2017analisis}). The presence of hills along bike routes pose two challenges for bike sharing systems. First, many commuters do not want to pedal up a steep hill on their morning or afternoon commute (\cite{midgley_2011}). It has been observed that many bike commuters prefer to choose a nonactive mode of transportation instead of biking up a steep hill (\cite{midgley_2011}). Second, many bike sharing stations experience an unequal distribution in bike availability, especially stations at higher elevations (\cite{midgley_2011}, \cite{HealthyRidePGHPSG}). Notably, the presence of steep inclinations along a route will increase the travel time for a bike commuter going uphill compared to a bike commuter traveling primarily downhill routes. Thus, it is expected that these two factors will play a significant role in how adults assess and evaluate the \textit{Manizales En Bici} bike sharing system.
\subsubsection*{Definition of Terms}
Below are definitions of terms that will be used throughout this study that may be associated with several definitions. To eliminate ambiguity, definitions of these terms are provided in the context that they will be referenced in this study. 
\begin{itemize}
    \item \underline{Smart Transportation}: Smart transportation, or smart mobility, in the context of this research study will be defined as transportation that is offered on demand and is presented as efficient, sustainable, flexible, and eco-friendly (\cite{SmartTransportation}).
    \item \underline{Adult}: An adult for this study includes any person of 18 years or older.
    \item \underline{Bike sharing system}: comprises short-term urban bicycle rental schemes that enable bicycles to be picked up at any self-serve bicycle station and returned to any other bicycle station; \textit{[Also may be referenced to as "public-use bicycles (PUBs)" or "Smart Bikes" throughout this paper]}. (\cite{midgley_2011})
    \item \underline{Infrastructure}: Infrastructure will be specifically referring to the bike sharing stations, the bicycles, any machine used to administer rental requests, and anything else specifically mentioned. 
    \item \underline{Initiatives}: Initiatives will be specifically relating to any policies, laws, regulations, project proposals, or ideas that have been put in place, have been discussed, or are set to be implemented in the future.
\end{itemize}

\subsubsection*{Limitations}
Limitations will be encountered while conducting this study. This is not a final cumulative list of all limitations that will be encountered and any limitation not addressed will be amended to the final draft. The following limitations should be kept in mind: 
\begin{itemize}
    \item This study will be conducted over a 6-week time frame in Manizales, Colombia in May 2021. 
    \item The study will be using convenience sampling to collect as many survey participants as possible that fit in the age range. Thus, the data may be skewed to which locations I spend my time at the most. 
\end{itemize}

\subsection*{Literature Review}
Manizales, Colombia is the capital of the department Caldas with a population of 400,436 people in 2018 and 71\% of the population being between 15 and 64 years old (\cite{CalidaddeVida}). In 2018, it was recorded that there were 445 vehicles per 1000 people which is equivalent of saying that there was one vehicle per every two people in Manizales (\cite{CalidaddeVida}). Despite the report of 445 vehicles per 1000 people, another model shows that 56\% of the population's main mode of transportation is by bus (\cite{CalidaddeVida}). This model also points out that only 12\% of the population reported their main mode of transportation to be walking or biking. To summarize here, the primary modes of transportation offered in Manizales, Colombia include public buses and busetas, personal vehicles, bicycles (including PUBs), walking, taxis, or the Cable Aereo (\cite{CalidaddeVida}). In terms of transportation infrastructure network, there is approximately 749km of road and the network that makes up the public bike sharing infrastructure is about 21km which is shared with other vehicles (\cite{montoya2017acceso}).

\subsubsection*{\textit{Manizales En Bici}}
Manizales, Colombia is one of many cities in Colombia that offer PUBs through a defined bike sharing program. Other cities in Colombia that offer a bike sharing program to the public include Medellín, Bogotá, Bucaramanga, and Villavicencio (\cite{montoya2017acceso}, \cite{PublicOpinPBS}). The public bike sharing system that Manizales, Colombia offers is called \textit{Manizales En Bici}. This system is provided to the public free of charge but they must register to be a user. The renting system works on biometrics where a finger print is required to access a bicycle; this rental system was invented by a company called CityBioBike (\cite{CityBioBike}) and works with the \textit{Manizales En Bici} bike sharing system. As of February 2020, currently 80 bikes of the 211 bikes and 26 electric assisted bike are available to the public through contract renewal between the Secretary of the Environment and the firm Dimat (\cite{lapatria_2020}). There are 8 stations throughout Manizales, Colombia that are equipped to host bicycles for public use through a rental process. These eight bike stations are located along a stretch of the same road that more or less runs horizontally through the entire city limits (\cite{zuluaga_garcía_2017}). CityBioBike released a Facebook post in the middle of March that provided statistics on the number of users they have recorded for their system (which is in collaboration with \textit{Manizales En Bici}); their post presents data for 2016 - 2019. As of the end of 2019, there was a recorded 7,692 users which is contrary to 7,431 users listed on Oficina de la bici official website (\cite{CityBioBike}. \cite{OficinaDeLaBiciHome}). Noticeably, there are several groups involved with the operations of the public bike sharing system in Manizales, Colombia. Thus, it is necessary to identify which areas and groups need improvement to accelerate the program. Having an in depth understanding of this system's users and its related infrastructure/initiatives is imperative to identifying which areas can use improvement for optimal success in Manizales. 

\subsubsection*{Related Bike Sharing Systems}
Ever since the United Nations released the seventeen sustainable development goals (SDGs) for the world, countries have been consistently integrating these goals into initiatives and projects within their cities. Improving transportation systems’ design and infrastructure can lead to be an overwhelming task accompanied with an exorbitant cost, but in the long run the benefits will distinctly present themselves. Smart transportation initiatives specifically relating to biking and bike sharing systems are growing rapidly within cities. Some cities, as Deliotte points out, are lacking in infrastructure to foster a well-established commuter population comprised of bikers (\cite{DeliotteReport}). Deliotte acknowledges from a study they did on smart mobility across cities in the United States that, “Slightly more than a quarter of current commuters could switch to biking as one of their main modes of commuting if barriers 
to biking were substantially reduced” (\cite{DeliotteReport}).
These barriers mentioned significantly affect the use and demand of a bike sharing system in any given city and 
there are several variables that contribute to this (\cite{ReviewonBike-sharing}). Focusing on other cities around that world of similar size to that of Manizales, Colombia that offer bike sharing systems, future possible improvements to "Manizales En Bici" will be derived. Two cities with two different types of terrain and bike sharing systems will be analyzed to derive hypothetical realistic improvements to \textit{Manizales En Bici}. The two cities that will be analyzed are Pittsburgh, Pennsylvania, United States and St. Louis, Missouri, United States. 

\underline{Pittsburgh, PA, USA}: Pittsburgh is the third largest city in Pennsylvania with a population of 301,000 people as of February 2020 (\cite{worldpopulationreviewPGH}). Pittsburgh features several hills and steep, uneven streets just like terrain seen in Manizales, Colombia. The resident bike sharing program in Pittsburgh is called HealthyRidePGH with 50 stations and 500 bikes dispersed throughout the city (\cite{HealthyRidePGHPSG}) and was introduced in 2015. The HealthyRide bike sharing programs deals with a higher amount of bikes at stations with lower elevations than those with higher elevations (\cite{HealthyRidePGHPSG}). Since there is no research on how the public views and evaluates the bike sharing system in Manizales, availability (reliability) is one of the key factors I will be analyzing. Furthermore, the HealthyRide bike sharing system costs money for a user to rent the bike. A user must download a smart phone application to set up an account and have a valid credit or debit card to pay the rental fee. This fee hinders the infrastructure from being damaged or stolen. On the contrary, \textit{Manizales En Bici} is a free system. Hence, I will seek to identify what percentage of respondents (if they don't use the system) are aware that it is free to use. The smart phone app also identifies which stations are nearby, where they are located, and if there are any bikes available. This aspect plays a role in the efficiency of the system and it is an aspect that \textit{Manizales En Bici} lacks. Therefore, the concept of having technology help identify bike availability is used as a hypothetical realistic improvement that is present to the respondents. 


\underline{St. Louis, MO, USA}: St. Louis is located along the Mississippi River on the boarder of Missouri and Illinois with a population of approximately 302,900 people as of February 2020 (\cite{worldpopulationreviewSTL}). St. Louis is characterized to feature a more flat terrain compared to Manizales and Pittsburgh. St. Louis coordinates with the company Lime to host their bike sharing services throughout the city. This is unique as it is an outside, for-profit company that is handling the services for the city. The city does need to subscribe to this service from Lime as they are a private, for-profit company. A unique difference about St. Louis's bike sharing program is the type of system being offered: a dock-less bike sharing system. This type of system does not require the user to return the bike to any given station allowing more freedom and feasibility on the individual. With this system, Lime reported in their yearly tech report in 2018 that they totaled over 60,000 unique riders (\cite{LimeBike}). Take note again the population size of St. Louis and Manizales. With such success in St. Louis, this type of system is being presented to users as a hypothetical realistic improvement to \textit{Manizales En Bici}. Another strong feature to note about the unique bike sharing iteration in St. Louis is the infrastructure created for the lower income population. By design, Lime services require the user to download the Lime smart phone app to locate and rent their bikes, but in collaboration with the city of St. Louis officials, they created a non-smart phone option, a non-credit card option, discounted rental fees, and deployed several of their bikes in low-income neighborhoods (\cite{gold_2018}). These solutions to provide a low-income community with a reliable transportation service is beneficial to the society as reliable transportation can allow an individual to obtain a reliable job and boost the overall economy. Some of these ideas that Lime implemented in St. Louis are highlighted when asking respondents how their evaluation of the system would change given new improvements. 

\subsubsection*{Smart Transportation \& Socio-Technical Systems}

Urban planners are constantly deriving and evaluating new strategies to satisfy desires for futuristic and sustainable transportation systems. A common characteristic that distinguishes cities with smart transportation systems is their initiatives to grow as a smart city that fosters sustainable development in other aspects than just transportation. The transition from a traditional transportation system in a city that prioritizes cars to a multifaceted, technology-enhanced transportation system can be understood through the notion of socio-technical systems (\cite{SmartTransportation}). A smart city's socio-technical system is crafted in a way such that citizens, software, hardware, data, and policies are all connected (\cite{rangwala_2018}). Bike sharing programs by their nature are labeled as "smart" because of the various applications of technology they feature. Being aware that the biking sharing system itself is not the sole contributor to success or failure is important while conducting this study. It is necessary to understand any technology, data, or regulations that play a role in the development of \textit{Manizales En Bici} and how the community responds. Another important attribute besides technology that fuels a successful bike sharing program is data, whether generated through crowd-sourcing techniques or automatically. User data that shows length of bike trip, starting and ending destination, and time of day are all important variables that need to be considered evaluating a bike sharing program. Visualization of this data can connect the community closer to factors that they take into consideration, but aren't aware of, when evaluating and assessing the state and quality of their city's bike sharing program.

\subsubsection*{Related Research \& Surveys}
The impacts of a bike sharing system in an urban environment have been proven to bring economic savings and a better quality of life to the community (\cite{DeliotteReport}, \cite{zuluaga_garcía_2017}, \cite{midgley_2011}, \cite{dávalos_maldonado_polit_2016}). It can also be correlated that bicycle commuting results in health benefits, decreased noise and air pollution, and decreased traffic congestion in urban centers (\cite{DeliotteReport}). These scientific research conclusions only can fuel the success of a public bike sharing system so far. There are few publicized research studies that discretely analyze and present the direct opinion of bicycle commuters vs non bicycle commuters. Furthermore, there is little published research that targets the opinions of adults who live in a city that hosts a bike sharing program. Despite the lack of research dedicated to analyzing the opinions related to the topic, one recent study has conducted a survey to analyze the main reasons why an individual uses or doesn't use the city's bike sharing system (\cite{PublicOpinPBS}). In fact, their survey was conducted in Villavicencio which is another city in Colombia. Their survey extracts a wide variety of reasons from respondents as to why they do or do not use the public bike sharing system. Their survey presented several options relating to weather, cost, skill set, location, or 'other'; the most popular answer remained 'other' among all questions. While their survey seeks the public opinion of users and non users of the public bike sharing system, it does not analyze what factors of the public bike sharing system are doing well or poor. It focuses solely on the users preference. I propose a survey that challenges the respondents to reflect on how they evaluate the public bike sharing system instead of solely evaluating their personal preferences. Personal preference is implicitly shown through their answers, but the survey questionnaires require the respondent to analyze the quality and state of the system. The research done in Villavicencio is very beneficial as it was conducted in 2017 and it provides a comparative case study. The survey questions can be used to guide how I will frame and present my survey questions.

Other results from a company called Lime, who is a leading company in the market for micro-mobility solutions in urban centers, presents in their one year report a select few statistics that highlight opinions from Lime users in select cities such as Seattle, Washington, USA or San Francisco, California, USA (\cite{LimeBike}). These results show what percentage of respondents use their service (Lime Bikes or Lime Scooters) to satisfy their first/last mile or what their destinations are. This report also identifies which mode of transportation the user would have most likely used if Lime was not offered to them. In San Francisco, 53\% of the respondents said that they would have taken a car. This report highlights many positive aspects of the service being offered. It is beneficial to understand which aspects are performing well, but there was little to no explicit presentation of areas for improvement. The statistics presented in this report are useful in terms of what aspects of a public bike sharing system I should take into consideration when asking users and non-users how they evaluate it themselves. 

A research project which was followed through as a Master's Thesis presents a direct case study for the public bike sharing system in Manizales, Colombia. This is the closet research publication I have found directed related to \textit{Manizales En Bici} thus far. Zuluaga identified that all 8 current \textit{Manizales En Bici} stations remain located in along the same road throughout Manizales (\cite{zuluaga_garcía_2017}). The basis of the research in this case study was to evaluate the current coverage (in 2017) of the public bike sharing stations in relation to the users of the system using ArcGIS mapping techniques. One insightful conclusion from this study identified that the locations of the \textit{Manizales En Bici} stations were more accessible or available to individual of higher socioeconomic status. Another conclusion included a recommendation of new station locations to improve the access and usage of the system throughout the community. While the locations of the stations are one of the dominant factors in the success of the bike sharing system overall, it is not enough to say the location and number of stations is the only area for improvement. Zuluaga agrees with this suggesting that preference studies should be conducted to understand how the city would react to hypothetical, yet realistic improvements to the public bike sharing system (\cite{zuluaga_garcía_2017}). This is the steering motive behind the format of my study: to perform a deep preference study that gather information of how the public assess the public bike system \textit{Manizales En Bici} and how they might assess the system given hypothetical and realistic future improvements. 

There is another research study that presents Manizales, Colombia as a case study. This study's main purpose was to determine minimum travel times to get around the city in comparison to socioeconomic status and geography (\cite{cardona}). The motivation behind this study came from a noticeable lack of infrastructure for a city that had a bike sharing program and an increase in traffic and population density (\cite{cardona}). Using different software than the last study, the grade of hills were determined and an algorithm was used to determine the travel time from one location to the next by bicycle. The study concluded the on average traveling by bicycle would result in similar travel times as traveling by car or bus, but that overall Manizales can be distinguished as a cycling city. It also concluded that individuals associated within the higher socioeconomic classes had an unfair advantage. Again, this is another technical research paper that presents the bicycle atmosphere in Manizales, Colombia as a case study. It highlights a socioeconomic disparity amongst biking within Manizales, as did the previous study mentioned (\cite{zuluaga_garcía_2017}). These two researchers, Zuluaga and Cardona, have produced a lot of research regarding \textit{Manizales En Bici} in regards to accessibility analyses and geographical contexts (\cite{cardona2017analisis}), but still no studies present an analysis of the public opinion on how individuals evaluate or assess \textit{Manizales En Bici} bike sharing program in Manizales.

\subsection*{Methodology}

A wide variety of methods will be used to gather data for this study. This study will gather qualitative and quantitative data from questionnaires that will be handed out by convenience sampling to obtain as many responses as possible within the 6-week time frame.

As previously mentioned, this study will be conducted through convenience sampling methods which could lead to error or bias in the resulting data. Those trends are yet to be determined. Participants of the survey will be 18 years or older. Because this study will be conducted using convenience sampling, participants will most likely come from the town center, multiple universities within the city of Manizales, bus stops, grocery stores, \textit{Manizales En Bici} bike sharing stations, and other common public locations. The goal is to receive as many responses as possible to analyze the data. The surveys will be printed on paper and will be handed to individuals that I approach in public places throughout the six week time frame.

\subsubsection*{Questionnaires}

The questionnaire will be split into three sections to better understand the participant’s perspective on the \textit{Manizales En Bici} public bike sharing system. The first section is to gather general information about the participant. General information includes age, gender, and what the most important factor to them is with respect to transportation. The factors presented in this question have been chosen through factors that are deemed vital to a successful bike sharing system (\cite{midgley_2011}). The themes of the factors throughout the entire survey are centered around safety, efficiency, and reliability. 

The second section goes in more depth about the current bike sharing system in Manizales, Colombia. This part of the questionnaire diverges into more specific questions depending on whether the respondent has ever used the bike sharing system before and if they are a frequent user. The first half of this section identifies if the respondent is familiar with \textit{Manizales En Bici} and if they have ever used it. If the respondent identifies that they are not a frequent user, they will be asked to identify the main reasons of why they are not a frequent user. The reasons that the respondents can choose from were created from conclusions of prior research (\cite{ReviewonBike-sharing}). Respondents who are frequent users will be asked to evaluate/rank how the bike sharing system is doing in several categories presented to them. 

The third section will address future hypothetical and realistic improvements to the \textit{Manizales En Bici} system. The purpose of this section is to encourage the respondents to critically think and reflect how might their assessment of the public bike sharing system change given certain improvements in certain areas relating to the systems infrastructure, technology, and safety. The third section is also used to ask respondents who have never used the public bike sharing system to identify why they have never used it based off of the presented factors. Some questions will require the participant to answer based on a scale of 0-5. The rating sections will be designed to better understand the factors that are used in the evaluation of \textit{Manizales En Bici}.

\subsubsection*{Collection \& Analysis of Data}

Qualitative data that is collected from the participants will be derived from all three sections. In the first section: gender and most important factor when it comes to transportation. In the second section qualitative data that will be gathered includes whether a respondent has ever heard of \textit{Manizales En Bici}, if they have ever used it, if they are a frequent user (and why they don't use it if they are not a frequent user), their main reasons for using \textit{Manizales En Bici}, if they live within a ten minute walk from one of the bike stations, and the ratings for each aspect related to the state or quality of \textit{Manizales En Bici} along with any comments or suggestions on how \textit{Manizales En Bici} can improve. In the third section: reasons for not using \textit{Manizales En Bici}, a question related to the likelihood of using \textit{Manizales En Bici} in the future given certain improvements.any comments or suggestions about how \textit{Manizales En Bici} can improve. Quantitative data that is collected from the participants will be derived from the first and second section. In the first section: age. In the second section: how many times the respondent uses \textit{Manizales En Bici} on average per week.
\subsubsection*{Potential Bias}
This study will be conducted through convenience sampling, thus there will be some bias in the results. The bias will have a trend on which areas I tend to interact with the most people along with the age of the people that tend to hang out in that area. 
\newpage
\printbibliography
\end{document}